\documentclass[a4paper,11pt]{article}
\usepackage{amsmath,amssymb,amsfonts,latexsym}
\usepackage[utf8]{inputenc}
\usepackage{fancyvrb}
\usepackage{graphicx}
\usepackage[margin=1.1in]{geometry}
\usepackage{url}
\usepackage{verbatim}
\usepackage{fixltx2e}
\usepackage{amsfonts}
\usepackage{amssymb}
\usepackage{amsmath}
\usepackage{float}
\usepackage{listings}
\usepackage{lmodern}
\usepackage[spanish]{babel}
\lstset{frame=tb,
  language=Java,
  aboveskip=3mm,
  belowskip=3mm,
  showstringspaces=false,
  columns=flexible,
  basicstyle={\small\ttfamily},
  numbers=none,
  breaklines=true,
  breakatwhitespace=true
  tabsize=3
}
\title{Speech compression \\
M\'etodos Num\'ericos Avanzados}
\author{Enzo Altamiranda, Cristian Ontiver, Valeria Serber}
\date{\today}
\begin{document}

\maketitle
\thispagestyle{empty}
\vspace{3cm}

\renewcommand{\abstractname}{Resumen}
\begin{abstract}
El siguiente informe explora el uso de la transformada rápida de Fourier, en
combinación con la codificación de Huffman para la compresión de voz. Se
realiza un estudio comparativo de diferentes niveles de cuantificación y sus
efectos tanto en el tamaño como la calidad - en términos de distorción - del
archivo comprimido.
\end{abstract}
\newpage
\section{Palabras Clave}
Transformada Rápida de Fourier, Compresión de voz, Codificación de Huffman
\newpage
\section{Introducci\'on}
\begin{comment}
[Introduce el tema contextualizando la informacion. Puede incluirse un parrafo
con una breve descripcion historica, otro parrafo motivando el tema. El ultimo
parrafo de esta seccion tiene que ser la descripcion de la estructura del
artıculo, explicitando en que seccion se trata cada tema.]
\end{comment}
\paragraph{}
La voz, entendida como el sonido generado por el aparato fonador humano, se
encuentra en frencuencias de entre $80$ a $1100$ Hz.  Teniendo esto en cuenta,
a la hora de almacenar digitalmente una señal de voz, es posible utilizar
métodos para reducir el tamaño necesario para guardar la información
correspondiente a esta señal.\\
Tradicionalmente estos se dividen en dos. Por un lado métodos "lossless" (sin
pérdida), es decir, métodos que almacenan la información de modo tal que esta
pueda ser recuperada en su totalidad. Por otro, métodos "lossy" (con pérdida),
en los cuales, tomando ventaja de que el rango de audición humano suele
situarse entre los $20$ y $20.000$ Hz (con variaciones de individuo a
individuo), descartan información considerada redundante (aquella por encima o
por debajo de ese rango).\\
Aquí exploramos una forma de lograr lo segundo, mediante el uso de la
transformada rápida de Fourier, en conjunto con la codificación de Huffman,
para lograr almacenar diferentes archivos de sonidos conteniendo voces, de
manera que se reduzca el tamaño necesario.\\
Como se verá más adelante, una reducción en la distorición y aumento en la
fidelidad del audio comprimido implican la necesidad de mayor información, y
por ende, una menor reducción en relación al tamaño (en bytes) original.
\paragraph{}
\newpage
\section{Metodolog\'ia}
\begin{comment}
[Hay que relatar los pasos que se fueron realizando, incluyendo los modelos
utilizados, los analisis hechos, las pruebas realizadas.]
\end{comment}
\subsubsection{Primera Iteraci\'on}
\section{Conclusiones}
\subsection{C\'alculo de A}
\paragraph{}
Si bien se logr\'o optimizar bastante el algoritmo que calcula la matriz A,
mientras se realizaban pruebas se pudo observar un patr\'on en la
construcci\'on de A. Se not\'o que a partir de \emph{m = 3}, la matriz \emph{A}
puede escribirse gen\'ericamente, ya que pequeños bloques de elementos se
repiten a lo largo de la estructura, aumentando la cantidad de bloques
l\'ogicamente mientras \emph{m} aumenta. Por lo tanto, se podr\'ia evitar tener
que calcular \emph{A} a partir del producto entre \emph{K} y \emph{L}, y en vez
de eso, seguir la siguiente regla para representarla:


\begin{equation} \label{matrizM4}
A =
\left( \begin{array}{cccccccccccc}
T & Q & R &   &   &  &  &  &  &  &  & S\\
  & S & T & Q & R &  &  &  &  &  &  &  \\
  &   &   & S & T & \ddots&  &  &  &  &  &  \\
  &   &   &   &   &  & \ddots&  &  &  &  &  \\
  &   &   &   &   &  &  & Q & R &  &  &  \\
  &   &   &   &   &  &  & S & T & Q & R &  \\
R &   &   &   &   &  &  &   &   & S & T & Q\\
\end{array} \right)
\end{equation}
\paragraph{}
Siendo
\begin{equation} \label{matrizQ}
Q =
\left( \begin{array}{cccc}
\sin \alpha & \cos \beta\\
\cos \alpha & \cos \beta\\
\end{array} \right)
\end{equation}

\begin{equation} \label{matrizR}
R =
\left( \begin{array}{cccc}
\sin \alpha & \sin \beta\\
\cos \alpha & \sin \beta\\
\end{array} \right)
\end{equation}

\begin{equation} \label{matrizS}
S =
\left( \begin{array}{cccc}
-\cos \alpha & \sin \beta\\
\sin \alpha & \sin \beta\\
\end{array} \right)
\end{equation}

\begin{equation} \label{matrizT}
T =
\left( \begin{array}{cccc}
\cos \alpha & \cos \beta\\
-\sin \alpha & \cos \beta\\
\end{array} \right)
\end{equation}

\newpage
\section{Bibliograf\'ia}
G. Rajesh, A. Kumar, K. Ranjeet. Speech Compression using Different Transform Techniques.
\end{document}
